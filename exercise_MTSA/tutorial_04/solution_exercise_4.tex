\documentclass[12pt,a4paper]{article}
\usepackage{lmodern}

\usepackage{placeins}
\usepackage{amssymb,amsmath}
\usepackage{ifxetex,ifluatex}
\usepackage{fixltx2e} % provides \textsubscript
\ifnum 0\ifxetex 1\fi\ifluatex 1\fi=0 % if pdftex
  \usepackage[T1]{fontenc}
  \usepackage[utf8]{inputenc}
\else % if luatex or xelatex
  \ifxetex
    \usepackage{mathspec}
    \usepackage{xltxtra,xunicode}
  \else
    \usepackage{fontspec}
  \fi
  \defaultfontfeatures{Mapping=tex-text,Scale=MatchLowercase}
  \newcommand{\euro}{€}
\fi
% use upquote if available, for straight quotes in verbatim environments
\IfFileExists{upquote.sty}{\usepackage{upquote}}{}
% use microtype if available
\IfFileExists{microtype.sty}{%
\usepackage{microtype}
\UseMicrotypeSet[protrusion]{basicmath} % disable protrusion for tt fonts
}{}
\usepackage[lmargin = 2cm, rmargin = 2cm, tmargin = 2cm, bmargin = 2.5cm]{geometry}


% Figure Placement:
\usepackage{float}
\let\origfigure\figure
\let\endorigfigure\endfigure
\renewenvironment{figure}[1][2] {
    \expandafter\origfigure\expandafter[H]
} {
    \endorigfigure
}

%%%% Jens %%%%
\usepackage{titlesec}
\DeclareMathOperator*{\argmax}{arg\,max}
\DeclareMathOperator*{\argmin}{arg\,min}
\renewcommand{\vec}{\operatorname{vec}}
\newcommand{\tr}{\operatorname{tr}}
\newcommand{\Var}{\operatorname{Var}} % Variance
\newcommand{\VAR}{\operatorname{VAR}} % Vector autoregression
\renewcommand{\L}{\operatorname{L}} % Lag operator
\newcommand{\Lag}{\operatorname{L}} % Lag operator
\newcommand{\Cov}{\operatorname{Cov}}
\newcommand{\diag}{\operatorname{diag}}
\newcommand{\adj}{\operatorname{adj}}

\allowdisplaybreaks

\titleformat{\section}
{\normalfont\large\bfseries}{\thesection}{1em}{}

\newcommand{\tmpsection}[1]{}
\let\tmpsection=\section
\renewcommand{\section}[1]{\tmpsection{\underline{#1}} }





%% citation setup
\usepackage{csquotes}

\usepackage[backend=biber, maxbibnames = 99, style = apa]{biblatex}
\setlength\bibitemsep{1.5\itemsep}
\addbibresource{R_packages.bib}
\usepackage{graphicx}
\makeatletter
\def\maxwidth{\ifdim\Gin@nat@width>\linewidth\linewidth\else\Gin@nat@width\fi}
\def\maxheight{\ifdim\Gin@nat@height>\textheight\textheight\else\Gin@nat@height\fi}
\makeatother
% Scale images if necessary, so that they will not overflow the page
% margins by default, and it is still possible to overwrite the defaults
% using explicit options in \includegraphics[width, height, ...]{}
\setkeys{Gin}{width=\maxwidth,height=\maxheight,keepaspectratio}
\ifxetex
  \usepackage[setpagesize=false, % page size defined by xetex
              unicode=false, % unicode breaks when used with xetex
              xetex]{hyperref}
\else
  \usepackage[unicode=true, linktocpage = TRUE]{hyperref}
\fi
\hypersetup{breaklinks=true,
            bookmarks=true,
            pdfauthor={Dr.~Yannick Hoga},
            pdftitle={Multivariate Time Series Analysis},
            colorlinks=true,
            citecolor=black,
            urlcolor=black,
            linkcolor=black,
            pdfborder={0 0 0}}
\urlstyle{same}  % don't use monospace font for urls
\setlength{\parindent}{0pt}
\setlength{\parskip}{6pt plus 2pt minus 1pt}
\setlength{\emergencystretch}{3em}  % prevent overfull lines
\setcounter{secnumdepth}{5}

%%% Use protect on footnotes to avoid problems with footnotes in titles
\let\rmarkdownfootnote\footnote%
\def\footnote{\protect\rmarkdownfootnote}

%%% Change title format to be more compact
\usepackage{titling}

% Create subtitle command for use in maketitle
\newcommand{\subtitle}[1]{
  \posttitle{
    \begin{center}\large#1\end{center}
    }
}

\setlength{\droptitle}{-2em}
  \title{Multivariate Time Series Analysis}
  \pretitle{\vspace{\droptitle}\centering\huge}
  \posttitle{\par}
\subtitle{Solution Exercise Sheet 4}
  \author{Dr.~Yannick Hoga}
  \preauthor{\centering\large\emph}
  \postauthor{\par}
  \date{}
  \predate{}\postdate{}


%% linespread settings

\usepackage{setspace}

\onehalfspacing


% Language Setup

\usepackage{ifthen}
\usepackage{iflang}
\usepackage[super]{nth}
\usepackage[ngerman, english]{babel}

%Acronyms
\usepackage[printonlyused, withpage, nohyperlinks]{acronym}
\usepackage{changepage}

% Multicols for the Title page
\usepackage{multicol}


% foot


\begin{document}

\selectlanguage{english}

%%%%%%%%%%%%%% Jens %%%%%
\numberwithin{equation}{section}




\restoregeometry


%%% Header 

\begin{minipage}{0.6\textwidth}
University of Duisburg-Essen\\
Faculty of Business Administration and Economics\\
Chair of Econometrics\\
\end{minipage}

%\begin{minipage}{0.4\textwidth}
	\begin{flushright}
	\vspace{-3cm}
	\includegraphics*[width=5cm]{../Includes/duelogo_en.png}\\
	\vspace{.125cm}
	\end{flushright}
%\end{minipage}
%\vspace{.125cm}
\hspace{-0.005cm}Winter Term 2019/2020

\vspace{0.05cm}

\begin{center}
	\vspace{.25cm}
	Dr.~Yannick Hoga \hspace{.5cm} Thilo Reinschlüssel \\
	\vspace{.25cm}
	\textbf{\Large{Multivariate Time Series Analysis}}\\
	\vspace{.25cm}
	\textbf{\large{Solution Exercise Sheet 4}}\\
	\vspace{.125cm}
\end{center}




% body from markdown

\hypertarget{exercise-1-implied-models-for-components}{%
\section{Exercise 1: Implied Models for
Components}\label{exercise-1-implied-models-for-components}}

Consider the VAR(1) model \(z_t = \phi_0 + \phi_1 z_{t-1} + a_t\) from
the Exercise Sheet 3 again:

\begin{align*}
    \phi_0 = \begin{pmatrix} 1 \\ 0 \end{pmatrix}, \quad \phi_1 = \begin{pmatrix} 0.75 & 0 \\ -0.25 & 0.5 \end{pmatrix}, \quad \Sigma_a = \begin{pmatrix} 1 & 0 \\ 0 & 1 \end{pmatrix} \ \text{.}
\end{align*}

\begin{itemize}
    \item[a)] Write down the model using lag operator notation. Then rearrange the equation such that
all parts based on $z_t$ are on the left-hand side and the remainder is on the right-hand side. 
\end{itemize}

\emph{Solution:}

\begin{align*}
  \text{Model:} z_t & = \phi_0 + \phi_1 z_{t-1} + a_t\\  
  \text{Lag notation:} \ z_t & = \phi_0 + \phi_1 \Lag z_t + a_t\\
  \\
  \Leftrightarrow z_t - \phi_1 \Lag Z_t & = \phi_0 + a_t
\end{align*}

\begin{itemize}
    \item[b)] By factoring out $z_t$ on the left, we obtain the lag polynomial $\phi (\Lag)$. Compute its adjoint matrix by hand.
    \item[]  \textit{Hint: Treat the lag operator as some scalar. The adjoint matrix can be computed like the inverse matrix but \textbf{without} the scaling by}  $\dfrac{1}{\det (\phi (\Lag))} \text{.}$
\end{itemize}

\emph{Solution:}

\begin{align*}
  \underbrace{\left( I - \phi_1 \Lag \right)}_{=: \phi \Lag} z_t & = \phi_0 + a_t\\
  \Leftrightarrow \phi (\Lag) & = \begin{pmatrix} 1- 0.75 \Lag & 0 \\ -0.25 \Lag & 1 - 0.5 \Lag  \end{pmatrix}\\
  \Leftrightarrow  \phi^{\adj} & = \begin{pmatrix} 1- 0.5 \Lag & 0 \\ 0.25 \Lag & 1 - 0.75 \Lag  \end{pmatrix}\\
\end{align*}

\begin{itemize}
    \item[c)]Pre-multiply the model equation you got in part a) with the adjoint matrix you computed in part b).
    \item[]  \textit{Hint: You are supposed to end up with a diagonal matrix.}
\end{itemize}

\emph{Solution:}

\begin{tiny}
\begin{align*}
  \begin{pmatrix} 1- 0.5 \Lag & 0 \\ 0.25 \Lag & 1 - 0.75 \Lag  \end{pmatrix} \begin{pmatrix} 1- 0.75 \Lag & 0 \\ -0.25 \Lag & 1 - 0.5 \Lag  \end{pmatrix} z_t & = \begin{pmatrix} 1 - 0.75 \Lag & 0 \\ 0.25 \Lag & 1 - 0.75 \Lag \end{pmatrix} \cdot \left( \phi_0 + a_t \right) \\
  \Leftrightarrow \begin{pmatrix} (1- 0.5 \Lag) (1- 0.75 \Lag) & 0 \\
  (0.25 \Lag) (1 - 0.75 \Lag) + (1 - 0.75 \Lag) (-0.25 \Lag) & (1 - 0.75 \Lag) (1 - 0.5 \Lag)
  \end{pmatrix} z_t  & = 
  \begin{pmatrix} (1 - 0.5 \Lag) \cdot 1 \\ 0.25 \Lag \cdot 1 +(1- 0.75 \Lag) \cdot 0 \end{pmatrix} +  
  \begin{pmatrix} (1 - 0.5 \Lag) \cdot a_{1,t} \\ (0.25 \Lag ) a_{1,t} + 1 (1- 0.75 \Lag) a_{2,t}\end{pmatrix} \\
  \Leftrightarrow \begin{pmatrix} z_{1,t} - 1.25 z_{1, t-1} + 0.375 z_{1, t-2} \\ z_{2,t} - 1.25 z_{2, t-1} + 0.375 z_{2, t-2} \end{pmatrix} 
  & = \begin{pmatrix} 0.5 \\ 0.25 \end{pmatrix} + \begin{pmatrix} a_{1,t} - 0.5 a_{1, t-1} \\ 
  0 + 0.25 a_{1, t-1} + a_{2,t} - 0.75 a_{2, t-1}\end{pmatrix}  
\end{align*}
\end{tiny}

\begin{itemize}
    \item[d)] The result of part c) should be a collection of two univariate ARMA(p,q) models. What is the lag order of both models?
\end{itemize}

\emph{Solution:}

\end{document}
