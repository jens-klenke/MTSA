\documentclass[12pt,a4paper]{article}
\usepackage{lmodern}

\usepackage{scalerel}
\usepackage{enumitem}
\usepackage{placeins}
\usepackage{amssymb,amsmath}
\usepackage{ifxetex,ifluatex}
\usepackage{fixltx2e} % provides \textsubscript
\ifnum 0\ifxetex 1\fi\ifluatex 1\fi=0 % if pdftex
  \usepackage[T1]{fontenc}
  \usepackage[utf8]{inputenc}
\else % if luatex or xelatex
  \ifxetex
    \usepackage{mathspec}
    \usepackage{xltxtra,xunicode}
  \else
    \usepackage{fontspec}
  \fi
  \defaultfontfeatures{Mapping=tex-text,Scale=MatchLowercase}
  \newcommand{\euro}{€}
\fi
% use upquote if available, for straight quotes in verbatim environments
\IfFileExists{upquote.sty}{\usepackage{upquote}}{}
% use microtype if available
\IfFileExists{microtype.sty}{%
\usepackage{microtype}
\UseMicrotypeSet[protrusion]{basicmath} % disable protrusion for tt fonts
}{}
\usepackage[lmargin = 2cm, rmargin = 2cm, tmargin = 2cm, bmargin = 2.5cm]{geometry}


% Figure Placement:
\usepackage{float}
\let\origfigure\figure
\let\endorigfigure\endfigure
\renewenvironment{figure}[1][2] {
    \expandafter\origfigure\expandafter[H]
} {
    \endorigfigure
}

%%%% Jens %%%%
\usepackage{titlesec}
\DeclareMathOperator*{\argmax}{arg\,max}
\DeclareMathOperator*{\argmin}{arg\,min}
\renewcommand{\vec}{\operatorname{vec}}
\newcommand{\tr}{\operatorname{tr}}
\newcommand{\Var}{\operatorname{Var}} % Variance
\newcommand{\MSE}{\operatorname{MSE}} % Variance
\newcommand{\VAR}{\operatorname{VAR}} % Vector autoregression
\newcommand{\Lag}{\operatorname{L}} % Lag operator
\newcommand{\Cov}{\operatorname{Cov}}
\newcommand{\diag}{\operatorname{diag}}
\newcommand{\adj}{\operatorname{adj}}
\newcommand{\loglik}{\operatorname{ll}}

\usepackage{centernot}

\allowdisplaybreaks

\titleformat{\section}
{\normalfont\large\bfseries}{\thesection}{1em}{}

\newcommand{\tmpsection}[1]{}
\let\tmpsection=\section
\renewcommand{\section}[1]{\tmpsection{\underline{#1}} }





%% citation setup
\usepackage{csquotes}

\usepackage[backend=biber, maxbibnames = 99, style = apa]{biblatex}
\setlength\bibitemsep{1.5\itemsep}
\addbibresource{R_packages.bib}
\usepackage{graphicx}
\makeatletter
\def\maxwidth{\ifdim\Gin@nat@width>\linewidth\linewidth\else\Gin@nat@width\fi}
\def\maxheight{\ifdim\Gin@nat@height>\textheight\textheight\else\Gin@nat@height\fi}
\makeatother
% Scale images if necessary, so that they will not overflow the page
% margins by default, and it is still possible to overwrite the defaults
% using explicit options in \includegraphics[width, height, ...]{}
\setkeys{Gin}{width=\maxwidth,height=\maxheight,keepaspectratio}
\ifxetex
  \usepackage[setpagesize=false, % page size defined by xetex
              unicode=false, % unicode breaks when used with xetex
              xetex]{hyperref}
\else
  \usepackage[unicode=true, linktocpage = TRUE]{hyperref}
\fi
\hypersetup{breaklinks=true,
            bookmarks=true,
            pdfauthor={Dr.~Yannick Hoga},
            pdftitle={Multivariate Time Series Analysis},
            colorlinks=true,
            citecolor=black,
            urlcolor=black,
            linkcolor=black,
            pdfborder={0 0 0}}
\urlstyle{same}  % don't use monospace font for urls
\setlength{\parindent}{0pt}
\setlength{\parskip}{6pt plus 2pt minus 1pt}
\setlength{\emergencystretch}{3em}  % prevent overfull lines
\setcounter{secnumdepth}{5}

%%% Use protect on footnotes to avoid problems with footnotes in titles
\let\rmarkdownfootnote\footnote%
\def\footnote{\protect\rmarkdownfootnote}

%%% Change title format to be more compact
\usepackage{titling}

% Create subtitle command for use in maketitle
\newcommand{\subtitle}[1]{
  \posttitle{
    \begin{center}\large#1\end{center}
    }
}

\setlength{\droptitle}{-2em}
  \title{Multivariate Time Series Analysis}
  \pretitle{\vspace{\droptitle}\centering\huge}
  \posttitle{\par}
\subtitle{Solution Exercise Sheet 7}
  \author{Dr.~Yannick Hoga}
  \preauthor{\centering\large\emph}
  \postauthor{\par}
  \date{}
  \predate{}\postdate{}


%% linespread settings

\usepackage{setspace}

\onehalfspacing


% Language Setup

\usepackage{ifthen}
\usepackage{iflang}
\usepackage[super]{nth}
\usepackage[ngerman, english]{babel}

%Acronyms
\usepackage[printonlyused, withpage, nohyperlinks]{acronym}
\usepackage{changepage}

% Multicols for the Title page
\usepackage{multicol}


% foot


\begin{document}

\selectlanguage{english}

%%%%%%%%%%%%%% Jens %%%%%
\numberwithin{equation}{section}




\restoregeometry


%%% Header 

\begin{minipage}{0.6\textwidth}
University of Duisburg-Essen\\
Faculty of Business Administration and Economics\\
Chair of Econometrics\\
\end{minipage}

%\begin{minipage}{0.4\textwidth}
	\begin{flushright}
	\vspace{-3cm}
	\includegraphics*[width=5cm]{../Includes/duelogo_en.png}\\
	\vspace{.125cm}
	\end{flushright}
%\end{minipage}
%\vspace{.125cm}
\hspace{-0.005cm}Winter Term 2019/2020

\vspace{0.05cm}

\begin{center}
	\vspace{.25cm}
	Dr.~Yannick Hoga \hspace{.5cm} Thilo Reinschlüssel \\
	\vspace{.25cm}
	\textbf{\Large{Multivariate Time Series Analysis}}\\
	\vspace{.25cm}
	\textbf{\large{Solution Exercise Sheet 7}}\\
	\vspace{.125cm}
\end{center}




% body from markdown

\hypertarget{exercise-1-the-optimal-forecast}{%
\section{Exercise 1: The optimal
forecast}\label{exercise-1-the-optimal-forecast}}

\begin{itemize}
  \item[a)] Show that the stationary $\VAR(1)$ process $z_t = t_{t-1} + a_t$ with $a_t$ a standard white noise has the following causal representation:
  
  $$z_t = \sum_{i  = 0}^{\infty} \theta_i a_{t-i} $$
\end{itemize}

\emph{Solution:}

\begin{align*}
  z_t & = \phi \cdot \underbrace{z_{t-1}}_{\phi z_{t-2} + a_{t-1}} + a_t \\
  & = \phi^2 z_{t-2} + \phi a_{t-1} + a_t\\
  & = \phi^3 z_{t-3} + \phi^2 a_{t-2} + \phi a_{t-1} + a_t\\
  & \vdots \\
  & = \phi^m z_{t -m} + \sum_{i = 0}^{m - 1} \phi^i a_{t-i}\\
  & = 0 + \sum_{i = 0}^{m - 1} \phi^i a_{t-i}˜ \\
  & \text{with} \ \lim_{m \rightarrow \infty} \phi^m = 0 \ \text{by weak stationarity}\\
  & = \sum_{i = 0}^{\infty} \theta_i a_{t-i}
\end{align*}

Using lag notation:

\begin{align*}
  z_t & = \phi  L z_t + a_t \\
  \Leftrightarrow (1 - \phi L)z_t & = a_t \\
  \Leftrightarrow z_t & = (1 - \phi L)^{-1} a_t
  \text{and} \quad (1 - \phi L)^{-1} & = \sum_{i = 0}^{\infty} \phi^i L^i \\
  & \text{(requires stationarity and invertiability)}
\end{align*}

\begin{itemize}
  \item[b)] Assume the linear forecasting model $y_T (h) = \psi y_T$ ans show that $\psi = \phi^h$ minimieses the $\MSE$ of $y_T (h)$  given that $y_t$ is a $\VAR(1)$ process. 
\end{itemize}

\emph{Solution:}

\begin{align*}
  y_T (h) = \argmin \underbrace{\MSE(y_T (h))}_{\mathbb{E} \left( \left[y_{T+h} - y_T (h) \right] \left[y_{T+h} - y_T (h) \right]^{'} \right)}\\
\end{align*}\\
\begin{align*}  
  \rightarrow Y_{T+h} & = \phi Y_{T+ h -1} + a_{T+h}\\
  & \vdots\\
  & = \phi^h  Y_T + \sum_{i = 0}^{h -1} \phi^i a_{T+h-i}\\
  \\
  \Rightarrow y_{T+ h} - Y_T (h) & = \phi^h Y_T + \sum_{i = 0}^{h -1} \phi^i a_{T + h -1} - \psi y_T\\
  \Rightarrow \MSE(y_T (h)) & = \mathbb{E} \left[ \underbrace{\left( \sum_{i = 0}^{h-1} \phi^i a_{T+h-i}  \right) \left( \sum_{i = 0}^{h-1} \phi^i a_{T+h-i}  \right)^{'} }_{\text{depends not on} \ \psi}       + \underbrace{\left(\phi^h - \psi \right)y_T y_T^{'} \left(\phi^h - \psi \right)}_{\text{minimised by} \ \psi = \phi^h} \right]
\end{align*}

\hypertarget{exercise-2-properties-of-forecast-errors}{%
\section{Exercise 2: Properties of forecast
errors}\label{exercise-2-properties-of-forecast-errors}}

\begin{itemize}
  \item[a)] Show that for a general $\VAR(p)$ process
  \[z_{T+h} - z_T (h) = e_T (h) = \sum_{i = 0}^{h -1} \theta_i a_{T + h -i}\]
  \begin{itemize}
    \item[] where $z_T (h)$ is assumed to be the optimal forecast.
  \end{itemize}
\end{itemize}

\emph{Solution:}

\begin{align*}
  z_T  & =  \phi_0 + \phi_1 z_{T-1} + \ldots  + \phi_p z_{T-p} + a_t \\
  \text{and forecast:} \ \ z_{T-1} (1) & = \phi_0  +\phi_1 z_{T-1} + \ldots  + \phi_p z_{T - p}\\
  \\
  \Rightarrow Z_{T + 1} - z_T (1) & = a_{T +1} = e_T (1) \\
  \Rightarrow z_{T + 2} - z_T (2) & = \phi_1 \underbrace{\left( z_{T + 1} - z_T (1) \right)} + a_{T + 2}\\
  & = \phi_1 a_{T+1} + a_{T+2}\\
  \Rightarrow z_{T + 3} - z_T (3) & = \phi_1 \underbrace{\left( Z_{T + 2} - z_{T} (2) \right)}_{\phi_1 a_{T + 1} + a_{T + 2}} + \phi_2 \underbrace{\left( z_{T +1} - Z_T (1) \right)}_{a_{T + 2}} + a_{T + 3} \\
  & = \phi_1 \left( \phi_1 a_{T + 1} + a_{T + 2} \right) + \phi_2 a_{T + 2} + a_{T + 3}\\
  & = \underbrace{\left( \phi_1^2 + \phi_2 \right)}_{\theta_2} a_{T + 1} + \underbrace{\phi_1}_{\theta_1} a_{T + 2} + \underbrace{ \ }_{I} a_{T + 3}\\
  & = \theta_2 a_{T + 1} + \theta_1 a_{T + 2} + \underbrace{\theta_0}_{I} a_{T + 3}\\
  \Rightarrow z_{T + h} - z_T (h) & = \theta_{h - 1} a_{T + 1} + \theta_{h - 2} a_{T + 2} + \ldots + \theta_{1} a_{T + h - 1} + \underbrace{\theta_0}_{I} a_{T + h}
\end{align*}

\begin{itemize}
  \item[b)] Assume that $a_t \sim N \left( 0, \Sigma_a \right)$. Derive the distribution of $e_T (h)$.   
\end{itemize}

\emph{Solution:} \begin{align*}
  \mathbb{E} \left[ z_{T +h}  - z_T (h) \right] & = \sum_{i = 0}^{h-1} \theta_i  \underbrace{\mathbb{E} \left[ a_{T + h - 1}\right]}_{= 0} = 0\\
  \\
  \Cov \left[ z_{T+} - z_T (h) \right]  & = \mathbb{E} \left[ \left( \sum_{i = 0}^{h-1} \theta_i a_{T + h - i} \right) \left( \sum_{j = 0}^{h-1} \theta_j a_{T + h - j} \right)^{'} \right]\\
   & = \mathbb{E} \left[ \sum_{i = 0}^{h-1} \theta_i \ a_{T + h - i} \ a_{T + h - j}^{'} \ \theta_i^{'} \right]\\
   & = \sum_{i = 0}^{h-1} \theta_i \ \mathbb{E} \left[\ a_{T + h - i} \ a_{T + h - j}^{'} \right] \ \theta_i^{'} \\
   & = \sum_{i = 0}^{h-1} \theta_i \ \Sigma_a  \theta_i^{'}  = \Sigma_e (h) \\
   & \text{since} \mathbb{E} \left( a_{T + h - i} a_{T + h - j} \right) = 0 \text{if} j \neq i 
\end{align*}

By using the fact that a sum of i.i.d. normally distributed variables
follows are normal distributed: \begin{align*}
  e_T (h) \sim \left( N, \Sigma_e (h)\right)\\
  \text{with} \ \Sigma_e (h) = \sum_{i = 0}^{h - 1} \theta_i \Sigma_a \theta_i
\end{align*}

\begin{itemize}
  \item[c)] Prove that $\Cov \left( e_T (h) \right) \rightarrow \Gamma_0$ as $h \rightarrow \infty$. 
\end{itemize}

\emph{Solution:}

\begin{align*}
  \lim_{h \rightarrow \infty} \Cov \left( e_T (h) \right) & = \mathbb{E} \left[ \left( \sum_{i = 0}^{\infty} \theta_i a_{T + h -i}\right)  \left( \sum_{i = 0}^{\infty} \theta_i a_{T + h -i}\right)^{'} \right] \\
  & = \lim_{h \rightarrow \infty} \mathbb{E} \left( z_{T + h} z_{T + h}^{'} \right)\\
  & = \lim_{h \rightarrow \infty} \Gamma_0^{T + h} = \Gamma_0\\
\end{align*}

by weak stationary.

\hypertarget{exercise-3-forecast-intervals}{%
\section{Exercise 3: Forecast
intervals}\label{exercise-3-forecast-intervals}}

\emph{Solution:}

\begin{align*}
  e_T (h) & \sim N \left( 0, \Cov(e_T(h))\right) \\
  & \text{For each element it holds that:}\\
  \dfrac{e_{T}^{(i)} (h) }{\sqrt{\Var(e_T^{(i)} (h))}} & \sim N(0,1)\\
  \\
  \Rightarrow & \text{we need} \ \sqrt{\Cov(e_T (h))} \\
  & \rotatebox[origin=c]{180}{$\Lsh$} \ \text{Cholesky decomposition of a positive deinifite matrix} \ A \\
  A & = UDU^{'} \text{, with D a diagonal matrix and U a lower traingular matrix}\\
  & \text{D can be split further into} \ D^{\frac{1}{2}} \cdot D^{\frac{1}{2}} \ (\text{note that} \ D^{'} = D \  \text{and} \ D^{\frac{1}{2}^{'}} = D^{\frac{1}{2}}) \\
  \Rightarrow A & = U D^{\frac{1}{2}} D^{\frac{1}{2}^{'}} = U D^{\frac{1}{2}} \left(U D^{\frac{1}{2}} \right)^{'}\\ 
  & = L L^{'}\\ 
  & \ \text{using} \ \Cov(e_T(h)) = \Cov(e_T(h))^{'} \ \text{by symmetry}
\end{align*}

\begin{align*}
  \Rightarrow e_T (h)^{'} \cdot \Cov(e_T(h))^{-1} \cdot e_T (h) & = e_T (h)^{'} \underbrace{\Cov(e_T(h))^{- \frac{1}{2}^{'}}}_{= L} \underbrace{\Cov(e_T(h))^{- \frac{1}{2}}}_{= L^{'}} e_T (h)\\
  & = \underbrace{\left( e_T (h) \ \Cov(e_T(h))^{- \frac{1}{2}} \right)^{'}}_{\sim N(0, I)} \underbrace{\left( e_T (h) \ \Cov(e_T(h))^{- \frac{1}{2}} \right)}_{\sim N(0, I)} 
\end{align*}

Both distributions are multivariate with \(K\) variables. Due to the
inner product we have a sum of \(K\) squared standard normal variables.

\(\Rightarrow\) this follows are \(\chi^2_K\) distribution!

The ellipsid an then be set up:

\[\left\{ z \in \mathbb{R}^K: e_T (h)^{'} \Cov(e_T (h))^{-1} e_T (h) \leq \chi_{K, 1- \alpha}^2 \right\}\]

\hypertarget{exercise-4-delta-method}{%
\section{Exercise 4: Delta Method}\label{exercise-4-delta-method}}

For this task, assume both \(y_T\) and \(x_t\) to be \(K \times 1\)
vectors and
\(x_t \overset{i.i.d.}{\sim} \left[ \mu_x, \Sigma_x\right]\).

\begin{itemize}
  \item[a)] Let $y_t = f(x_t) = \phi_1 x_t$. Compute the mean and variance of $y_T$.
\end{itemize}

\emph{Solution:}

\begin{align*}
  y_t & = \phi_1 x_t\\
  \\
  \mathbb{E} & = \mathbb{E} (\phi_1 x_t) = \phi_1 \mathbb{E} (x_t) = \phi_1 \mu_x \\
  y_t - \mathbb{E}(y_t) & = \tilde{y}_t = \phi_1 (x_t - \mu_x) = \phi_1 \tilde{x}_t\\
  \\
  \Cov(y_t) & = \Cov(\tilde{y}_t) = \Cov(\phi_1 \tilde{x}_t) \\
  & = \phi_1 \mathbb{E} (\tilde{x}_t \tilde{x}_t^{'}) \phi_1^{'} = \phi_1 \Sigma_x \phi_1^{'}
\end{align*}

\begin{itemize}
  \item[b)] Derive the distribution of $\sqrt{T} \left(\bar{y_T} - E(y) \right)$ from your results in a). 
\end{itemize}

\emph{Solution:}

\begin{align*}
  x_t \  \text{ is i.i.d. distributed, } \mathbb{E}(x_t)< \infty, \Cov(x_t) < \infty \\
  \\
  \Rightarrow \text{ a CLT applies!}\\
  \sqrt{T} \left( \bar{Y}_T - \mathbb{E} (y) \right) \overset{d}{\longrightarrow} N(0, \phi_1 \Sigma_x \phi_1^{'} )
\end{align*}

\begin{itemize}
  \item[c)] Now let $f(\cdot)$ be some function $f(x): \mathbb{R}^K \mapsto \mathbb{R}^K$. Derive the first order Taylor expansion for $f(x)$ at $\mu_x$ and write it down in detail.  
\end{itemize}

\emph{Solution:}

\begin{align*}
  f(x) & = f(x_1, \ldots, f_k)\\
  & = \begin{pmatrix}
    f_1 (x_1, \ldots, f_k)\\
    \vdots \\
    f_k (x_1, \ldots, f_k)
  \end{pmatrix}\\
  \\
  & \text{1st order Tylor expansion:}\\
  \\
  f(x) & \approx f (\mu_x) + \begin{pmatrix}
    \frac{\partial f_1}{\partial x_1} & \cdots & \frac{\partial f_1}{\partial x_k}\\
    \vdots \\
    \frac{\partial f_x}{\partial x_1} & \cdots & \frac{\partial f_k}{\partial x_k}\\  
    \end{pmatrix} \cdot 
    \begin{pmatrix}
      x_1 - \mu_{x_1}^{(1)}\\
      \vdots \\
      x_k - \mu_{x_k}^{(k)}
    \end{pmatrix}
\end{align*}

\begin{itemize}
  \item[d)] Based on the expression in c), show that a CLT applies for $\sqrt{T} \left( f \left(\bar{X_T} \right) - f \left( \mu_x \right) \right)$, and derive the distribution. 
\end{itemize}

\emph{Solution:}

\begin{align*}
  \text{Use} \; x & = \bar{x_T}\\
  \Rightarrow f(\bar{x_T}) & \approx f(\mu_x) + J (\bar{x_T} - \mu_x)\\
  \Leftrightarrow f(\bar{x_T} - f(\mu_x)) & \approx J (\bar{x_T} - \mu_x)\\
  \mathbb{E} \left( f \left( \bar{x_T}\right) - f \left( \mu_x\right) \right) & = \mathbb{E} \left( J \cdot \bar{x_t} - \mu_x \right)\\
  & = J \cdot 0 = 0 \\
  \\
  \Rightarrow \Cov \left(  f \left( \bar{x_T}\right) - f \left( \mu_x\right) \right) & = \mathbb{E} \left( \left( f \left( \bar{x_T}\right) - f \left( \mu_x\right) \right) \left(  f \left( \bar{x_T}\right) - f \left( \mu_x\right) \right)^{'}  \right) \\
  & = \mathbb{E} \left( J  \left( \tilde{x_T} - \mu_x \right) \left( \tilde{x_T} - \mu_x \right)^{'} J^{'}  \right) \\
  & = J \ \mathbb{E} \left( \left( \dfrac{1}{T} \sum_{t = 1}^{T} \tilde{x_t} \right) \left( \dfrac{1}{T} \sum_{t = 1}^{T} \tilde{x_t} \right)^{'} \right) J\\
  & = J \ \mathbb{E} \left( \dfrac{1}{T} \left( \underbrace{\sum_{t = 1}^{T} \tilde{x_t} \tilde{x_t^{'}}}_{\Sigma_x} + \underbrace{\sum_{t \neq s}^{T} \tilde{x_t} \tilde{x_s^{'}}}_{E(\cdot) = 0 }  \right) \right) J \cdot \dfrac{1}{T}\\
  & = \dfrac{1}{T} J \cdot \Sigma_x \cdot J^{'}\\
  & \Rightarrow \sqrt{T} \left( f (\bar{x}_T - f(\mu_x)) \right) \overset{d}{\longrightarrow} N(0, J \Sigma_x J^{'})
\end{align*}

\begin{itemize}
  \item[e)] Lastly, assume the variable $x_t$ to be known (meaning it is not stochastic). We want to predict $y_t$ using $y_t = \phi_1 x_t$. Unfortunately, we only have $\widehat{\phi_1}$ which is stochastic with $\sqrt{T} \left( \widehat{\phi_1} - \phi_1 \right) \overset{d}{\longrightarrow} N(0, \Sigma_{\phi})$. Can we say something about the distribution of the prediction error $\hat{y}_t - y_t$?
\end{itemize}

\emph{Solution:}

\begin{align*}
  \hat{y_t} - y_t & = \underbrace{\left( \widehat{\phi_1} - \phi_1 \right)}_{\text{stochastic}} \underbrace{x_t}_{\text{deterministic}}\\
  \mathbb{E} \left( \widehat{y_t} - y_t \right) & = \underbrace{\mathbb{E} \left( \widehat{\phi_1} - \phi_1 \right) }_{= 0} x_t = 0 \\
  \\
  \Cov( f(\bar{x_T}) - f(\mu_x) ) & = \Cov \left( \left( \widehat{\phi_1} - \phi_1 \right) x_t \right) \\
  & = x_t^{'} \Cov \left( \widehat{\phi_1} - \phi_1 \right) x_t\\
  & =  x_t^{'} \Sigma_{\phi_1} x_t
\end{align*}

and since \(\widehat{\phi_1}\) follows a normal distribution:

\[\widehat{y_t} - y_t \sim N(0, x^{'} \Sigma_{\phi_1} x)\]

\end{document}
