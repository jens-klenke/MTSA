\documentclass[12pt,a4paper]{article}
\usepackage{lmodern}

\usepackage{scalerel}
\usepackage{bbm}
\usepackage{enumitem}
\usepackage{placeins}
\usepackage{amssymb,amsmath}
\usepackage{ifxetex,ifluatex}
\usepackage{fixltx2e} % provides \textsubscript
\ifnum 0\ifxetex 1\fi\ifluatex 1\fi=0 % if pdftex
  \usepackage[T1]{fontenc}
  \usepackage[utf8]{inputenc}
\else % if luatex or xelatex
  \ifxetex
    \usepackage{mathspec}
    \usepackage{xltxtra,xunicode}
  \else
    \usepackage{fontspec}
  \fi
  \defaultfontfeatures{Mapping=tex-text,Scale=MatchLowercase}
  \newcommand{\euro}{€}
\fi
% use upquote if available, for straight quotes in verbatim environments
\IfFileExists{upquote.sty}{\usepackage{upquote}}{}
% use microtype if available
\IfFileExists{microtype.sty}{%
\usepackage{microtype}
\UseMicrotypeSet[protrusion]{basicmath} % disable protrusion for tt fonts
}{}
\usepackage[lmargin = 2cm, rmargin = 2cm, tmargin = 2cm, bmargin = 2.5cm]{geometry}


% Figure Placement:
\usepackage{float}
\let\origfigure\figure
\let\endorigfigure\endfigure
\renewenvironment{figure}[1][2] {
    \expandafter\origfigure\expandafter[H]
} {
    \endorigfigure
}

%%%% Jens %%%%
\usepackage{titlesec}
\DeclareMathOperator*{\argmax}{arg\,max}
\DeclareMathOperator*{\argmin}{arg\,min}
\renewcommand{\vec}{\operatorname{vec}}
\newcommand{\tr}{\operatorname{tr}}
\newcommand{\Var}{\operatorname{Var}} % Variance
\newcommand{\MSE}{\operatorname{MSE}} % Variance
\newcommand{\VAR}{\operatorname{VAR}} % Vector autoregression
\newcommand{\Lag}{\operatorname{L}} % Lag operator
\newcommand{\Cov}{\operatorname{Cov}}
\newcommand{\diag}{\operatorname{diag}}
\newcommand{\adj}{\operatorname{adj}}
\newcommand{\loglik}{\operatorname{ll}}

\usepackage{centernot}
\usepackage{mathtools}

\allowdisplaybreaks

\titleformat{\section}
{\normalfont\large\bfseries}{\thesection}{1em}{}

\newcommand{\tmpsection}[1]{}
\let\tmpsection=\section
\renewcommand{\section}[1]{\tmpsection{\underline{#1}} }





%% citation setup
\usepackage{csquotes}

\usepackage[backend=biber, maxbibnames = 99, style = apa]{biblatex}
\setlength\bibitemsep{1.5\itemsep}
\addbibresource{R_packages.bib}
\usepackage{color}
\usepackage{fancyvrb}
\newcommand{\VerbBar}{|}
\newcommand{\VERB}{\Verb[commandchars=\\\{\}]}
\DefineVerbatimEnvironment{Highlighting}{Verbatim}{commandchars=\\\{\}}
% Add ',fontsize=\small' for more characters per line
\usepackage{framed}
\definecolor{shadecolor}{RGB}{248,248,248}
\newenvironment{Shaded}{\begin{snugshade}}{\end{snugshade}}
\newcommand{\AlertTok}[1]{\textcolor[rgb]{0.94,0.16,0.16}{#1}}
\newcommand{\AnnotationTok}[1]{\textcolor[rgb]{0.56,0.35,0.01}{\textbf{\textit{#1}}}}
\newcommand{\AttributeTok}[1]{\textcolor[rgb]{0.77,0.63,0.00}{#1}}
\newcommand{\BaseNTok}[1]{\textcolor[rgb]{0.00,0.00,0.81}{#1}}
\newcommand{\BuiltInTok}[1]{#1}
\newcommand{\CharTok}[1]{\textcolor[rgb]{0.31,0.60,0.02}{#1}}
\newcommand{\CommentTok}[1]{\textcolor[rgb]{0.56,0.35,0.01}{\textit{#1}}}
\newcommand{\CommentVarTok}[1]{\textcolor[rgb]{0.56,0.35,0.01}{\textbf{\textit{#1}}}}
\newcommand{\ConstantTok}[1]{\textcolor[rgb]{0.00,0.00,0.00}{#1}}
\newcommand{\ControlFlowTok}[1]{\textcolor[rgb]{0.13,0.29,0.53}{\textbf{#1}}}
\newcommand{\DataTypeTok}[1]{\textcolor[rgb]{0.13,0.29,0.53}{#1}}
\newcommand{\DecValTok}[1]{\textcolor[rgb]{0.00,0.00,0.81}{#1}}
\newcommand{\DocumentationTok}[1]{\textcolor[rgb]{0.56,0.35,0.01}{\textbf{\textit{#1}}}}
\newcommand{\ErrorTok}[1]{\textcolor[rgb]{0.64,0.00,0.00}{\textbf{#1}}}
\newcommand{\ExtensionTok}[1]{#1}
\newcommand{\FloatTok}[1]{\textcolor[rgb]{0.00,0.00,0.81}{#1}}
\newcommand{\FunctionTok}[1]{\textcolor[rgb]{0.00,0.00,0.00}{#1}}
\newcommand{\ImportTok}[1]{#1}
\newcommand{\InformationTok}[1]{\textcolor[rgb]{0.56,0.35,0.01}{\textbf{\textit{#1}}}}
\newcommand{\KeywordTok}[1]{\textcolor[rgb]{0.13,0.29,0.53}{\textbf{#1}}}
\newcommand{\NormalTok}[1]{#1}
\newcommand{\OperatorTok}[1]{\textcolor[rgb]{0.81,0.36,0.00}{\textbf{#1}}}
\newcommand{\OtherTok}[1]{\textcolor[rgb]{0.56,0.35,0.01}{#1}}
\newcommand{\PreprocessorTok}[1]{\textcolor[rgb]{0.56,0.35,0.01}{\textit{#1}}}
\newcommand{\RegionMarkerTok}[1]{#1}
\newcommand{\SpecialCharTok}[1]{\textcolor[rgb]{0.00,0.00,0.00}{#1}}
\newcommand{\SpecialStringTok}[1]{\textcolor[rgb]{0.31,0.60,0.02}{#1}}
\newcommand{\StringTok}[1]{\textcolor[rgb]{0.31,0.60,0.02}{#1}}
\newcommand{\VariableTok}[1]{\textcolor[rgb]{0.00,0.00,0.00}{#1}}
\newcommand{\VerbatimStringTok}[1]{\textcolor[rgb]{0.31,0.60,0.02}{#1}}
\newcommand{\WarningTok}[1]{\textcolor[rgb]{0.56,0.35,0.01}{\textbf{\textit{#1}}}}
\usepackage{graphicx}
\makeatletter
\def\maxwidth{\ifdim\Gin@nat@width>\linewidth\linewidth\else\Gin@nat@width\fi}
\def\maxheight{\ifdim\Gin@nat@height>\textheight\textheight\else\Gin@nat@height\fi}
\makeatother
% Scale images if necessary, so that they will not overflow the page
% margins by default, and it is still possible to overwrite the defaults
% using explicit options in \includegraphics[width, height, ...]{}
\setkeys{Gin}{width=\maxwidth,height=\maxheight,keepaspectratio}
\ifxetex
  \usepackage[setpagesize=false, % page size defined by xetex
              unicode=false, % unicode breaks when used with xetex
              xetex]{hyperref}
\else
  \usepackage[unicode=true, linktocpage = TRUE]{hyperref}
\fi
\hypersetup{breaklinks=true,
            bookmarks=true,
            pdfauthor={Dr.~Yannick Hoga},
            pdftitle={Multivariate Time Series Analysis},
            colorlinks=true,
            citecolor=black,
            urlcolor=black,
            linkcolor=black,
            pdfborder={0 0 0}}
\urlstyle{same}  % don't use monospace font for urls
\setlength{\parindent}{0pt}
\setlength{\parskip}{6pt plus 2pt minus 1pt}
\setlength{\emergencystretch}{3em}  % prevent overfull lines
\setcounter{secnumdepth}{5}

%%% Use protect on footnotes to avoid problems with footnotes in titles
\let\rmarkdownfootnote\footnote%
\def\footnote{\protect\rmarkdownfootnote}

%%% Change title format to be more compact
\usepackage{titling}

% Create subtitle command for use in maketitle
\newcommand{\subtitle}[1]{
  \posttitle{
    \begin{center}\large#1\end{center}
    }
}

\setlength{\droptitle}{-2em}
  \title{Multivariate Time Series Analysis}
  \pretitle{\vspace{\droptitle}\centering\huge}
  \posttitle{\par}
\subtitle{Solution Exercise Sheet 8}
  \author{Dr.~Yannick Hoga}
  \preauthor{\centering\large\emph}
  \postauthor{\par}
  \date{}
  \predate{}\postdate{}


%% linespread settings

\usepackage{setspace}

\onehalfspacing


% Language Setup

\usepackage{ifthen}
\usepackage{iflang}
\usepackage[super]{nth}
\usepackage[ngerman, english]{babel}

%Acronyms
\usepackage[printonlyused, withpage, nohyperlinks]{acronym}
\usepackage{changepage}

% Multicols for the Title page
\usepackage{multicol}


% foot


\begin{document}

\selectlanguage{english}

%%%%%%%%%%%%%% Jens %%%%%
\numberwithin{equation}{section}




\restoregeometry


%%% Header 

\begin{minipage}{0.6\textwidth}
University of Duisburg-Essen\\
Faculty of Business Administration and Economics\\
Chair of Econometrics\\
\end{minipage}

%\begin{minipage}{0.4\textwidth}
	\begin{flushright}
	\vspace{-3cm}
	\includegraphics*[width=5cm]{../Includes/duelogo_en.png}\\
	\vspace{.125cm}
	\end{flushright}
%\end{minipage}
%\vspace{.125cm}
\hspace{-0.005cm}Winter Term 2019/2020

\vspace{0.05cm}

\begin{center}
	\vspace{.25cm}
	Dr.~Yannick Hoga \hspace{.5cm} Thilo Reinschlüssel \\
	\vspace{.25cm}
	\textbf{\Large{Multivariate Time Series Analysis}}\\
	\vspace{.25cm}
	\textbf{\large{Solution Exercise Sheet 8}}\\
	\vspace{.125cm}
\end{center}




% body from markdown

\hypertarget{exercise-1-forecast-intervals-and-distributional-assumptions}{%
\section{Exercise 1: Forecast Intervals and Distributional
Assumptions}\label{exercise-1-forecast-intervals-and-distributional-assumptions}}

\begin{itemize}
  \item[a)] Which key assumption about the innovations $a_t$ is made in the lecture to derive the distribution of the forecast errors $e_T (h)$? 
\end{itemize}

\emph{Solution:}

\begin{align*}
  a_t & \overset{\text{i.i.d.}}{\sim} N (0, \Sigma_a)
\end{align*}

\begin{itemize}
  \item i.i.d $\Rightarrow$ no autocorrelation
  \item $N \ \Rightarrow$ normal distributed 
  \item $\Sigma_a \ \Rightarrow$ heteroskedasticity
\end{itemize}

\begin{itemize}
  \item[b)] Assume we knew all parameters / coefficients and let $\Sigma_a$ be the identity matrix $I_{3 \times 3}$. Based on the assumption from a), derive the distribution of $e_T (1)$  for any stationary $\VAR(p)$.  
\end{itemize}

\emph{Solution:}

\begin{align*}
  e_T (1) & = z_{T + 1} - z_T (1) \\
  & = a_{T + 1}\\
  \\
  & \text{holds for any } \VAR(p) \ \text{since}\\
  z_T (1) & = \mathbb{E} \left(z_{T + 1} | z_T, \ldots , z_0 \right)\\
  \\
  \Rightarrow e_t (1) & = a_{T + 1} \sim N \left(0, I_{3 \times 3} \right)
\end{align*}

\begin{itemize}
  \item[c)] Derive the confidence ellipsoid associated to b) for $\alpha = 5 \%$. What is the fraction forecast errors that lie inside the ellposid? 
\end{itemize}

\emph{Solution:}

llipsoid:

\[\left\{ z \in \mathbb{R}^{3}: \left( z_T (1) - z \right)^{'} \Sigma_e^{-1} (1) \left( z_T (1) - z \right) \leq \chi_{3, 1- \alpha}^{2}\right\}\]

By defining \(z_T (1) - z =: \epsilon\) and using that
\(\Sigma_e (1) = \Sigma_a = I_{3 \times 3}\) the ellipsoid is:
\(\left\{ \epsilon \in \mathbb{R}^{3}: \epsilon^{'} \ \epsilon \leq \chi_{3, 1- \alpha}^{2}\right\}\).
For \(\alpha = 5 \%\), \(95 \%\) of the observed forecast errors are
expected to fall inside the confidence ellipsoid.

\begin{itemize}
  \item[d)] Run a simulation in `R `: Draw the forecast error $e_T (1)$ defined in a) and b) with $K = 3$. Check if it is located inside or outside the confidence ellipsoid derived in c). Use $N = 10 000$ repetitions in total and conclude whether the confidence ellipsoid is appropriate.  
\end{itemize}

\emph{Solution:}

Just check if \(e_T (1)^{'} e_T (1) \leq \chi_{3, 0.95}^{2}\) and
compute
\(\displaystyle \dfrac{1}{N} \sum_{i = 1}^{N} \mathbbm{1} \left( e_T (1)^{'} e_T (1) \leq \chi_{3, 0.95}^{2}\right)\).

\begin{Shaded}
\begin{Highlighting}[]
\NormalTok{N <-}\StringTok{ }\DecValTok{10000} \CommentTok{# number of repetitions}
\NormalTok{K <-}\StringTok{ }\DecValTok{3} \CommentTok{# dimension of VAR}
 \CommentTok{# drawing a_t from iid N(0,I)}
\NormalTok{gauss <-}\StringTok{ }\KeywordTok{mvrnorm}\NormalTok{(}\DataTypeTok{n =}\NormalTok{ N, }\DataTypeTok{mu =} \KeywordTok{c}\NormalTok{(}\DecValTok{0}\NormalTok{,}\DecValTok{0}\NormalTok{,}\DecValTok{0}\NormalTok{), }\DataTypeTok{Sigma =} \KeywordTok{diag}\NormalTok{(K))}
\CommentTok{# computing e'e for all draws in one take equals diag(EE')}
\NormalTok{msfe_gauss <-}\StringTok{ }\KeywordTok{diag}\NormalTok{(gauss }\OperatorTok\StringTok{ }\KeywordTok{t}\NormalTok{(gauss)) }
\CommentTok{# this is only a one-sided test since we have squared each error!}
\NormalTok{limit <-}\StringTok{ }\KeywordTok{qchisq}\NormalTok{(}\DataTypeTok{p =} \FloatTok{0.95}\NormalTok{, }\DataTypeTok{df=}\DecValTok{3}\NormalTok{, }\DataTypeTok{lower.tail=}\OtherTok{TRUE}\NormalTok{) }
\KeywordTok{sum}\NormalTok{(msfe_gauss }\OperatorTok{<}\StringTok{ }\NormalTok{limit) }\OperatorTok{/}\StringTok{ }\NormalTok{N}
\end{Highlighting}
\end{Shaded}

\begin{verbatim}
## [1] 0.9499
\end{verbatim}

\begin{itemize}
  \item[e)] Repeat the simulation from above but this time assume $a_t$ to be drawn from a uniform distribution. $\Sigma_a = I_{3 \times 3}$ remains. How reliable is the confidence ellipsoid in this case? 
  \textit{Hint: Set $\pm \dfrac{\sqrt{12}}{2}$ as lower / upper bound for unit variance.}
\end{itemize}

\emph{Solution:}

\begin{Shaded}
\begin{Highlighting}[]
\CommentTok{# variance = 1 again, Kurtosis is < 3 for this one}
\NormalTok{unif <-}\StringTok{ }\KeywordTok{matrix}\NormalTok{(}\DataTypeTok{data =} \KeywordTok{runif}\NormalTok{(}\DataTypeTok{n =}\NormalTok{ N }\OperatorTok{*}\StringTok{ }\DecValTok{3}\NormalTok{, }\DataTypeTok{min =} \OperatorTok{-}\KeywordTok{sqrt}\NormalTok{(}\DecValTok{12}\NormalTok{)}\OperatorTok{/}\DecValTok{2}\NormalTok{, }
                            \DataTypeTok{max =} \KeywordTok{sqrt}\NormalTok{(}\DecValTok{12}\NormalTok{)}\OperatorTok{/}\DecValTok{2}\NormalTok{), }\DataTypeTok{nrow =}\NormalTok{ N, }\DataTypeTok{ncol =} \DecValTok{3}\NormalTok{) }
\NormalTok{msfe_unif <-}\StringTok{ }\KeywordTok{rowSums}\NormalTok{(unif}\OperatorTok{^}\DecValTok{2}\NormalTok{)}
\KeywordTok{sum}\NormalTok{(msfe_unif }\OperatorTok{<}\StringTok{ }\NormalTok{limit) }\OperatorTok{/}\StringTok{ }\NormalTok{N}
\end{Highlighting}
\end{Shaded}

\begin{verbatim}
## [1] 0.9987
\end{verbatim}

It is to conservative.

\begin{itemize}
  \item[f)] Repeat the simulation drawing innovations from a $t$-distribution with $2$ degrees of freedom and conclude.
\end{itemize}

\emph{Solution:}

\begin{Shaded}
\begin{Highlighting}[]
\CommentTok{# variance = 1 by default, kurtosis > 3 and this hurts a lot}
\NormalTok{t2 <-}\StringTok{ }\KeywordTok{matrix}\NormalTok{(}\DataTypeTok{data =} \KeywordTok{rt}\NormalTok{(}\DataTypeTok{n =}\NormalTok{ N }\OperatorTok{*}\StringTok{ }\DecValTok{3}\NormalTok{, }\DataTypeTok{df =} \DecValTok{2}\NormalTok{), }\DataTypeTok{nrow =}\NormalTok{ N, }\DataTypeTok{ncol =} \DecValTok{3}\NormalTok{)}
\NormalTok{msfe_t2 <-}\StringTok{ }\KeywordTok{rowSums}\NormalTok{(t2}\OperatorTok{^}\DecValTok{2}\NormalTok{)}
\KeywordTok{sum}\NormalTok{(msfe_t2 }\OperatorTok{<}\StringTok{ }\NormalTok{limit) }\OperatorTok{/}\StringTok{ }\NormalTok{N}
\end{Highlighting}
\end{Shaded}

\begin{verbatim}
## [1] 0.6424
\end{verbatim}

Too liberal, the ellipsoid is not appropriate.

\end{document}
