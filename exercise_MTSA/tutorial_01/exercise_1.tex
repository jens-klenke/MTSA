\documentclass[12pt,a4paper]{article}
\usepackage{lmodern}

\usepackage{placeins}
\usepackage{amssymb,amsmath}
\usepackage{ifxetex,ifluatex}
\usepackage{fixltx2e} % provides \textsubscript
\ifnum 0\ifxetex 1\fi\ifluatex 1\fi=0 % if pdftex
  \usepackage[T1]{fontenc}
  \usepackage[utf8]{inputenc}
\else % if luatex or xelatex
  \ifxetex
    \usepackage{mathspec}
    \usepackage{xltxtra,xunicode}
  \else
    \usepackage{fontspec}
  \fi
  \defaultfontfeatures{Mapping=tex-text,Scale=MatchLowercase}
  \newcommand{\euro}{€}
\fi
% use upquote if available, for straight quotes in verbatim environments
\IfFileExists{upquote.sty}{\usepackage{upquote}}{}
% use microtype if available
\IfFileExists{microtype.sty}{%
\usepackage{microtype}
\UseMicrotypeSet[protrusion]{basicmath} % disable protrusion for tt fonts
}{}
\usepackage[lmargin = 2cm, rmargin = 2.5cm, tmargin = 2cm, bmargin = 2.5cm]{geometry}


% Figure Placement:
\usepackage{float}
\let\origfigure\figure
\let\endorigfigure\endfigure
\renewenvironment{figure}[1][2] {
    \expandafter\origfigure\expandafter[H]
} {
    \endorigfigure
}

%%%% Jens %%%%
\usepackage{titlesec}
\DeclareMathOperator*{\argmax}{arg\,max}
\DeclareMathOperator*{\argmin}{arg\,min}


\titleformat{\section}
{\normalfont\large\bfseries}{\thesection}{1em}{}

\newcommand{\tmpsection}[1]{}
\let\tmpsection=\section
\renewcommand{\section}[1]{\tmpsection{\underline{#1}} }


%% citation setup
\usepackage{csquotes}

\usepackage[backend=biber, maxbibnames = 99, style = apa]{biblatex}
\setlength\bibitemsep{1.5\itemsep}
\addbibresource{R_packages.bib}
\usepackage{graphicx}
\makeatletter
\def\maxwidth{\ifdim\Gin@nat@width>\linewidth\linewidth\else\Gin@nat@width\fi}
\def\maxheight{\ifdim\Gin@nat@height>\textheight\textheight\else\Gin@nat@height\fi}
\makeatother
% Scale images if necessary, so that they will not overflow the page
% margins by default, and it is still possible to overwrite the defaults
% using explicit options in \includegraphics[width, height, ...]{}
\setkeys{Gin}{width=\maxwidth,height=\maxheight,keepaspectratio}
\ifxetex
  \usepackage[setpagesize=false, % page size defined by xetex
              unicode=false, % unicode breaks when used with xetex
              xetex]{hyperref}
\else
  \usepackage[unicode=true, linktocpage = TRUE]{hyperref}
\fi
\hypersetup{breaklinks=true,
            bookmarks=true,
            pdfauthor={Dr.~Yannick Hoga},
            pdftitle={Multivariate Time Series Analysis},
            colorlinks=true,
            citecolor=black,
            urlcolor=black,
            linkcolor=black,
            pdfborder={0 0 0}}
\urlstyle{same}  % don't use monospace font for urls
\setlength{\parindent}{0pt}
\setlength{\parskip}{6pt plus 2pt minus 1pt}
\setlength{\emergencystretch}{3em}  % prevent overfull lines
\setcounter{secnumdepth}{5}

%%% Use protect on footnotes to avoid problems with footnotes in titles
\let\rmarkdownfootnote\footnote%
\def\footnote{\protect\rmarkdownfootnote}

%%% Change title format to be more compact
\usepackage{titling}

% Create subtitle command for use in maketitle
\newcommand{\subtitle}[1]{
  \posttitle{
    \begin{center}\large#1\end{center}
    }
}

\setlength{\droptitle}{-2em}
  \title{Multivariate Time Series Analysis}
  \pretitle{\vspace{\droptitle}\centering\huge}
  \posttitle{\par}
\subtitle{Exercise Sheet 1}
  \author{Dr.~Yannick Hoga}
  \preauthor{\centering\large\emph}
  \postauthor{\par}
  \date{}
  \predate{}\postdate{}

\usepackage{booktabs}
\usepackage{longtable}
\usepackage{array}
\usepackage{multirow}
\usepackage{wrapfig}
\usepackage{float}
\usepackage{colortbl}
\usepackage{pdflscape}
\usepackage{tabu}
\usepackage{threeparttable}
\usepackage{threeparttablex}
\usepackage[normalem]{ulem}
\usepackage{makecell}
\usepackage{xcolor}

%% linespread settings

\usepackage{setspace}

\onehalfspacing

% Language Setup

\usepackage{ifthen}
\usepackage{iflang}
\usepackage[super]{nth}
\usepackage[ngerman, english]{babel}

%Acronyms
\usepackage[printonlyused, withpage, nohyperlinks]{acronym}
\usepackage{changepage}

% Multicols for the Title page
\usepackage{multicol}

\begin{document}

\selectlanguage{english}

%%%%%%%%%%%%%% Jens %%%%%
\numberwithin{equation}{section}




\restoregeometry


%%% Header 

\begin{minipage}{0.6\textwidth}
University of Duisburg-Essen\\
Faculty of Business Administration and Economics\\
Chair of Econometrics\\
\end{minipage}

%\begin{minipage}{0.4\textwidth}
	\begin{flushright}
	\vspace{-3cm}
	\includegraphics*[width=5cm]{Includes/duelogo_en.png}\\
	\vspace{.5cm}
	\end{flushright}
%\end{minipage}
\vspace{.25cm}
\hspace{-0.005cm}Winter Term 2019/2020

\vspace{0.25cm}

\begin{center}
	\vspace{.25cm}
	Dr.~Yannick Hoga \hspace{.5cm} Thilo Reinschlüssel \\
	\vspace{.25cm}
	\textbf{\Large{Multivariate Time Series Analysis}}\\
	\vspace{.25cm}
	\textbf{\large{Exercise Sheet 1}}\\
	\vspace{.125cm}
\end{center}


% body from markdown

\hypertarget{exercise-1-matrix-operations}{%
\section{Exercise 1: Matrix
Operations}\label{exercise-1-matrix-operations}}

Prove properties 3,4 and 5 from Proposition 1.2 (Slide 1-11). Are there
any requirements regarding the matrix dimensions?

\emph{Solution:}\\

\begin{itemize}
    \item[i)] Property 3: $(A \otimes B)(F \otimes G) = (AF) \otimes (BG)$
    \begin{align*}
      \text{Let} \ A & = 
      \begin{pmatrix}
      a_{11} & \ldots & a_{1q}\\
      \vdots & \ddots & \vdots \\
      a_{p1} & \ldots & a_{pq}\\
      \end{pmatrix}
      \text{and} \ F = 
      \begin{pmatrix}
      f_{11} & \ldots & f_{1n}\\
      \vdots & \ddots & \vdots \\
      f_{m1} & \ldots & f_{mn}\\
      \end{pmatrix}\\
      \text{hence} (A \otimes B) & = 
      \begin{pmatrix}
      a_{11}B & \ldots & a_{1q}B\\
      \vdots & \ddots & \vdots \\
      a_{p1}B & \ldots & a_{pq}B\\
      \end{pmatrix}
      \text{and} \ (F \otimes G) \ \text{analsgously}
    \end{align*}
    \begin{align*}
    (A \otimes B)(F \otimes G) & = 
    \begin{pmatrix}
      a_{11}B & \ldots & a_{1q}B\\
      \vdots & \ddots & \vdots \\
      a_{p1}B & \ldots & a_{pq}B\\
    \end{pmatrix}
    \begin{pmatrix}
      f_{11}G & \ldots & f_{1n}G\\
      \vdots & \ddots & \vdots \\
      f_{m1}G & \ldots & f_{mn}G\\
    \end{pmatrix}\\
    & = 
    \begin{pmatrix}
      (a_{11}B f_{11}G +\ldots + a_{1q}B f_{m1}G) & \ldots & (a_{11}B f_{1n}G +\ldots + a_{1q}B f_{mn}G)\\
      \vdots & \ddots & \vdots \\
      (a_{p1}B f_{11}G +\ldots + a_{pq}B f_{m1}G) & \ldots & (a_{p1}B f_{1n}G +\ldots + a_{pq}B f_{mn}G)\\
    \end{pmatrix}\\
     & = 
    \begin{pmatrix}
      (a_{11} f_{11} +\ldots + a_{1q} f_{m1}) & \ldots & (a_{11} f_{1n} +\ldots + a_{1q} f_{mn})\\
      \vdots & \ddots & \vdots \\
      (a_{p1} f_{11} +\ldots + a_{pq} f_{m1}) & \ldots & (a_{p1} f_{1n} +\ldots + a_{pq} f_{mn})\\
    \end{pmatrix} \otimes (BG) \\
    \end{align*}
    \begin{align*}
    & = 
    \begin{pmatrix}
      \sum_{i = 1}^{q = m} a_{1i} f_{i1} & \ldots &  \sum_{i = 1}^{q = m} a_{1i} f_{1i} \\
      \vdots & \ddots & \vdots \\
      \sum_{i = 1}^{q = m} a_{pi} f_{i1} & \ldots & \sum_{i = 1}^{q = m} a_{pi} f_{in}\\
    \end{pmatrix} \otimes (BG) \qquad \quad \\
    & = (AF) \otimes (BG) 
    \end{align*}
    Dimensions: 
      \begin{table}[h]
      \centering
      \begin{tabular}{|l|l|}
        \hline
          $A: p \times q$ & $F: m \times n$ \\ \hline
          $B: c \times d$ & $G: h \times k$ \\ \hline
      \end{tabular}
      \end{table}
    \begin{itemize}
      \item[$\Rightarrow$] $dim(A \otimes B) = pc \times qd, dim(F \otimes G) = mh \times kn$
    \end{itemize}
    \item[ii)] Property 4: $(A \otimes B)^{-1} = A^{-1} \otimes B^{-1}$ \\
    $\Rightarrow$ Claim and verify \\
    The inverse is defined as following: \\
    $(A \otimes B)(A \otimes B)^{-1} = I$ where $I$ is the identitiy matrix\\
    Then $(A \otimes B)(A^{-1} \otimes B^{-1}) = I$ must hold if the claim was true\\
    We know from Property 3 that $(A \otimes B)(A^{-1} \otimes B^{-1}) = (AA^{-1} \otimes BB^{-1}) =  I \otimes I = I$ \\
    Dimensions: $A$ and $B$ must be non-singular square matrices
    \item[iii)] Property 3: $tr(A \otimes C ) = tr(A) \cdot tr( C)$ for square matrices $A$ and $C$
    \begin{align*}
      tr(A \otimes C) = 
      tr 
      \begin{pmatrix}
        a_{11}C & \ldots & a_{1n}C \\
        \vdots & \ddots & \vdots \\
        a_{n1}C & \ldots & a_{nn}C
      \end{pmatrix}
      = \sum_{i = 1}^{n}\left( a_{ii} tr(C) \right) = tr(C) \sum_{i=1}^{n} a_{ii} = tr(C) tr(A)
    \end{align*}
\end{itemize}

\hypertarget{excerise-2-bivariate-functions}{%
\section{Excerise 2: Bivariate
Functions}\label{excerise-2-bivariate-functions}}

Find the extrema of the following functions (using pen and paper).
Determine whether these points constitute minima, maxima or saddle
points:

\begin{itemize}
    \item[a)] $f(x,y) = (x -2)^2 + (y -5)^2 + xy$
    \item[b)] $g(x,y) = (x -1)^3 - (4y + 1)^2$
\end{itemize}

\emph{Solution:}\\
Soluton concept:

\begin{enumerate}
  \item FOC: first derivatives $\overset{!}{=} 0$
  \item SOC: check the determinant of the Hessian matrix
\end{enumerate}

\begin{itemize}
    \item[a)] $f(x,y) = (x -2)^2 + (y -5)^2 + xy$
    \begin{align*}
      f(x,y) & = (x -2)^2 + (y - 5)^2 + xy\\
      & \\
      \dfrac{\partial f(x,y)}{\partial x} & = 2(x -2) + y \overset{!}{=} 0
      \dfrac{\partial f(x,y)}{\partial y} & = 2(y -5) + x \overset{!}{=} 0 
    \end{align*}
    \begin{itemize}
      \item Solving the equation system yields:
    \end{itemize}
    \begin{align*}
      x = 2 -\dfrac{y}{2} \Rightarrow 2y - 10 + 2 - \dfrac{y}{2} = 0 &\Rightarrow y^{*} = \dfrac{16}{3} \\
      &\Rightarrow x^{*} = 2 -\dfrac{16}{3 \cdot 2} = - \dfrac{2}{3}
    \end{align*}
    \begin{itemize}
      \item Evaluting the Hessian matrix:
    \end{itemize}
    \begin{align*}
      \dfrac{\partial f(x,y)}{\partial x^2} = 2 \qquad  \dfrac{\partial f(x,y)}{\partial xy} = 1 \\
      \dfrac{\partial f(x,y)}{\partial yx} = 1  \qquad  \dfrac{\partial f(x,y)}{\partial y^2} = 2\\
      \Rightarrow H = 
      \begin{pmatrix}
        2 & 1 \\
        1 & 2
      \end{pmatrix}
    \end{align*}
    \begin{itemize}
      \item[] and $det(H) = 2 \cdot 2 - 1 \cdot 1 = 3 > 0 \;$ which indicates a minimum
    \end{itemize}
    \item[b)] $g(x,y) = (x -1)^3 - (4y + 1)^2$
    \begin{align*}
      g(x,y) & = (x -1)^3 + (4y - 1)^2\\
      & \\
      \dfrac{\partial g(x,y)}{\partial x} & = 3(x - 1)^2 \overset{!}{=} 0 \Leftrightarrow x^{*} = 1 \\
      \dfrac{\partial g(x,y)}{\partial y} & = 2 \cdot 4(4y -5) + x \overset{!}{=} 0 \Leftrightarrow y^{*} = - \dfrac{1}{4} 
    \end{align*}
    \begin{itemize}
      \item Evaluting the Hessian matrix:
    \end{itemize}
    \begin{align*}
      \dfrac{\partial g(x,y)}{\partial x^2} = 6x- 6 \qquad  \dfrac{\partial g(x,y)}{\partial xy} = 0 \\
      \dfrac{\partial g(x,y)}{\partial yx} = 0  \qquad  \dfrac{\partial g(x,y)}{\partial y^2} = 32\\
      \Rightarrow H = 
      \begin{pmatrix}
        6x- 6 & 0 \\
        0 & 32
      \end{pmatrix}
    \end{align*}
    \begin{itemize}
      \item[] and $det(H)|_{x = x^{*}, y = y^{*}} = (6-6)  \cdot 32 - 0 \cdot 0 = 0 \;$ which indicates a saddle point. Thus we did not find an extremal point.
    \end{itemize}
\end{itemize}

\hypertarget{excerise-3-stationarity}{%
\section{Excerise 3: Stationarity}\label{excerise-3-stationarity}}

\begin{itemize}
    \item[a)] Are weakly stationary processes always strictly stationary? Construct an example to support your argument
    \item[b)] Is weak stationarity a necessary condition for strict stationarity? Bring an example.
    \item[] \textit{Hint: How many moments does a distribution require?}
\end{itemize}

\emph{Solution:}

\begin{itemize}
    \item[a)] No. A time series of length $T$ drawing from $N(0, 1)$ for $t \in \left[0, \dfrac{T}{2} \right]$ and drawing from Student’s t-distribution for $t \in \left(\dfrac{T}{2}, T \right]$ has a constant mean $\mu = 0$ and variance $\sigma^2 = 1$, but the kurtosis ($4^{th}$ moment) changes throughout time. In consequence the joint distribution of a subsequence $x_{t-p}, \ldots , x_{t+p}$ is not independent of $t$. Therefore it is not strictly stationary
    \item[b)] No. Take the Cauchy distribution as an example: $f(x) = \dfrac{1}{\pi} \cdot \dfrac{s}{s^2 + (x -t)^2}$. Any $i.i.d.$ sample from this distribution would be obviously strictly stationary. Yet this distribution has no existing moments at all (the integral diverges), hence it cannot exhibit a constant expected value or variance over time. Therefore it is only strictly stationary, but not weakly stationary! (Other example: $t_1$ distribution, where only the mean but not the variance exists).
\end{itemize}

\hypertarget{excerise-3-covariance-matrices-under-stationarity}{%
\section{Excerise 3: Covariance Matrices under
Stationarity}\label{excerise-3-covariance-matrices-under-stationarity}}

Referring to Remark 1.13: Show that \(\Gamma_l = \Gamma^{T}_{-l}\) holds
for all weakly stationary processes.

(\emph{Two dimensions suffice})

\emph{Solution:}

Without loss of generality assume \(\mu = 0\) everywhere and assume
\(z\) to be a bivariate vector \((x,y)^T\). Let \(\Gamma_{l,t}\) be the
covariance matrix of the \(l^{th}\) lag at time \(t\):

\begin{align*}
\Gamma_{l,t} = 
\begin{bmatrix}
  \mathbb{E}(x_t \cdot x_{t-l}) & \mathbb{E}(x_t \cdot y_{t-l}) \\
  \mathbb{E}(y_t \cdot x_{t-l}) & \mathbb{E}(y_t \cdot y_{t-l}) \\
\end{bmatrix}
\quad \text{and} \quad 
\Gamma_{l,t}^{T} = 
\begin{bmatrix}
  \mathbb{E}(x_{t-l} \cdot x_t ) & \mathbb{E}(x_{t-l} \cdot y_t) \\
  \mathbb{E}(y_{t-l} \cdot x_t  ) & \mathbb{E}( y_{t-l} \cdot y_t) \\
\end{bmatrix}
= \Gamma_{-l, t- l}
\end{align*}

Since weak stationarity has been assumed, the covariance matrix is
constant across time and
\(\Gamma_{-l, t- l} = \Gamma_{-l} = \Gamma_{l}^{T}\) and vice versa.

\end{document}
