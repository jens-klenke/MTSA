\documentclass[12pt,a4paper]{article}
\usepackage{lmodern}

\usepackage{placeins}
\usepackage{amssymb,amsmath}
\usepackage{ifxetex,ifluatex}
\usepackage{fixltx2e} % provides \textsubscript
\ifnum 0\ifxetex 1\fi\ifluatex 1\fi=0 % if pdftex
  \usepackage[T1]{fontenc}
  \usepackage[utf8]{inputenc}
\else % if luatex or xelatex
  \ifxetex
    \usepackage{mathspec}
    \usepackage{xltxtra,xunicode}
  \else
    \usepackage{fontspec}
  \fi
  \defaultfontfeatures{Mapping=tex-text,Scale=MatchLowercase}
  \newcommand{\euro}{€}
\fi
% use upquote if available, for straight quotes in verbatim environments
\IfFileExists{upquote.sty}{\usepackage{upquote}}{}
% use microtype if available
\IfFileExists{microtype.sty}{%
\usepackage{microtype}
\UseMicrotypeSet[protrusion]{basicmath} % disable protrusion for tt fonts
}{}
\usepackage[lmargin = 2cm, rmargin = 2.5cm, tmargin = 2cm, bmargin = 2.5cm]{geometry}


% Figure Placement:
\usepackage{float}
\let\origfigure\figure
\let\endorigfigure\endfigure
\renewenvironment{figure}[1][2] {
    \expandafter\origfigure\expandafter[H]
} {
    \endorigfigure
}

%%%% Jens %%%%
\usepackage{titlesec}
\DeclareMathOperator*{\argmax}{arg\,max}
\DeclareMathOperator*{\argmin}{arg\,min}
\renewcommand{\vec}{\operatorname{vec}}

\titleformat{\section}
{\normalfont\large\bfseries}{\thesection}{1em}{}

\newcommand{\tmpsection}[1]{}
\let\tmpsection=\section
\renewcommand{\section}[1]{\tmpsection{\underline{#1}} }


%% citation setup
\usepackage{csquotes}

\usepackage[backend=biber, maxbibnames = 99, style = apa]{biblatex}
\setlength\bibitemsep{1.5\itemsep}
\addbibresource{R_packages.bib}
\usepackage{color}
\usepackage{fancyvrb}
\newcommand{\VerbBar}{|}
\newcommand{\VERB}{\Verb[commandchars=\\\{\}]}
\DefineVerbatimEnvironment{Highlighting}{Verbatim}{commandchars=\\\{\}}
% Add ',fontsize=\small' for more characters per line
\usepackage{framed}
\definecolor{shadecolor}{RGB}{248,248,248}
\newenvironment{Shaded}{\begin{snugshade}}{\end{snugshade}}
\newcommand{\AlertTok}[1]{\textcolor[rgb]{0.94,0.16,0.16}{#1}}
\newcommand{\AnnotationTok}[1]{\textcolor[rgb]{0.56,0.35,0.01}{\textbf{\textit{#1}}}}
\newcommand{\AttributeTok}[1]{\textcolor[rgb]{0.77,0.63,0.00}{#1}}
\newcommand{\BaseNTok}[1]{\textcolor[rgb]{0.00,0.00,0.81}{#1}}
\newcommand{\BuiltInTok}[1]{#1}
\newcommand{\CharTok}[1]{\textcolor[rgb]{0.31,0.60,0.02}{#1}}
\newcommand{\CommentTok}[1]{\textcolor[rgb]{0.56,0.35,0.01}{\textit{#1}}}
\newcommand{\CommentVarTok}[1]{\textcolor[rgb]{0.56,0.35,0.01}{\textbf{\textit{#1}}}}
\newcommand{\ConstantTok}[1]{\textcolor[rgb]{0.00,0.00,0.00}{#1}}
\newcommand{\ControlFlowTok}[1]{\textcolor[rgb]{0.13,0.29,0.53}{\textbf{#1}}}
\newcommand{\DataTypeTok}[1]{\textcolor[rgb]{0.13,0.29,0.53}{#1}}
\newcommand{\DecValTok}[1]{\textcolor[rgb]{0.00,0.00,0.81}{#1}}
\newcommand{\DocumentationTok}[1]{\textcolor[rgb]{0.56,0.35,0.01}{\textbf{\textit{#1}}}}
\newcommand{\ErrorTok}[1]{\textcolor[rgb]{0.64,0.00,0.00}{\textbf{#1}}}
\newcommand{\ExtensionTok}[1]{#1}
\newcommand{\FloatTok}[1]{\textcolor[rgb]{0.00,0.00,0.81}{#1}}
\newcommand{\FunctionTok}[1]{\textcolor[rgb]{0.00,0.00,0.00}{#1}}
\newcommand{\ImportTok}[1]{#1}
\newcommand{\InformationTok}[1]{\textcolor[rgb]{0.56,0.35,0.01}{\textbf{\textit{#1}}}}
\newcommand{\KeywordTok}[1]{\textcolor[rgb]{0.13,0.29,0.53}{\textbf{#1}}}
\newcommand{\NormalTok}[1]{#1}
\newcommand{\OperatorTok}[1]{\textcolor[rgb]{0.81,0.36,0.00}{\textbf{#1}}}
\newcommand{\OtherTok}[1]{\textcolor[rgb]{0.56,0.35,0.01}{#1}}
\newcommand{\PreprocessorTok}[1]{\textcolor[rgb]{0.56,0.35,0.01}{\textit{#1}}}
\newcommand{\RegionMarkerTok}[1]{#1}
\newcommand{\SpecialCharTok}[1]{\textcolor[rgb]{0.00,0.00,0.00}{#1}}
\newcommand{\SpecialStringTok}[1]{\textcolor[rgb]{0.31,0.60,0.02}{#1}}
\newcommand{\StringTok}[1]{\textcolor[rgb]{0.31,0.60,0.02}{#1}}
\newcommand{\VariableTok}[1]{\textcolor[rgb]{0.00,0.00,0.00}{#1}}
\newcommand{\VerbatimStringTok}[1]{\textcolor[rgb]{0.31,0.60,0.02}{#1}}
\newcommand{\WarningTok}[1]{\textcolor[rgb]{0.56,0.35,0.01}{\textbf{\textit{#1}}}}
\usepackage{graphicx}
\makeatletter
\def\maxwidth{\ifdim\Gin@nat@width>\linewidth\linewidth\else\Gin@nat@width\fi}
\def\maxheight{\ifdim\Gin@nat@height>\textheight\textheight\else\Gin@nat@height\fi}
\makeatother
% Scale images if necessary, so that they will not overflow the page
% margins by default, and it is still possible to overwrite the defaults
% using explicit options in \includegraphics[width, height, ...]{}
\setkeys{Gin}{width=\maxwidth,height=\maxheight,keepaspectratio}
\ifxetex
  \usepackage[setpagesize=false, % page size defined by xetex
              unicode=false, % unicode breaks when used with xetex
              xetex]{hyperref}
\else
  \usepackage[unicode=true, linktocpage = TRUE]{hyperref}
\fi
\hypersetup{breaklinks=true,
            bookmarks=true,
            pdfauthor={Dr.~Yannick Hoga},
            pdftitle={Multivariate Time Series Analysis},
            colorlinks=true,
            citecolor=black,
            urlcolor=black,
            linkcolor=black,
            pdfborder={0 0 0}}
\urlstyle{same}  % don't use monospace font for urls
\setlength{\parindent}{0pt}
\setlength{\parskip}{6pt plus 2pt minus 1pt}
\setlength{\emergencystretch}{3em}  % prevent overfull lines
\setcounter{secnumdepth}{5}

%%% Use protect on footnotes to avoid problems with footnotes in titles
\let\rmarkdownfootnote\footnote%
\def\footnote{\protect\rmarkdownfootnote}

%%% Change title format to be more compact
\usepackage{titling}

% Create subtitle command for use in maketitle
\newcommand{\subtitle}[1]{
  \posttitle{
    \begin{center}\large#1\end{center}
    }
}

\setlength{\droptitle}{-2em}
  \title{Multivariate Time Series Analysis}
  \pretitle{\vspace{\droptitle}\centering\huge}
  \posttitle{\par}
\subtitle{Exercise Sheet 2}
  \author{Dr.~Yannick Hoga}
  \preauthor{\centering\large\emph}
  \postauthor{\par}
  \date{}
  \predate{}\postdate{}


%% linespread settings

\usepackage{setspace}

\onehalfspacing


% Language Setup

\usepackage{ifthen}
\usepackage{iflang}
\usepackage[super]{nth}
\usepackage[ngerman, english]{babel}

%Acronyms
\usepackage[printonlyused, withpage, nohyperlinks]{acronym}
\usepackage{changepage}

% Multicols for the Title page
\usepackage{multicol}


% foot





\begin{document}

\selectlanguage{english}

%%%%%%%%%%%%%% Jens %%%%%
\numberwithin{equation}{section}




\restoregeometry


%%% Header 

\begin{minipage}{0.6\textwidth}
University of Duisburg-Essen\\
Faculty of Business Administration and Economics\\
Chair of Econometrics\\
\end{minipage}

%\begin{minipage}{0.4\textwidth}
	\begin{flushright}
	\vspace{-3cm}
	\includegraphics*[width=5cm]{Includes/duelogo_en.png}\\
	\vspace{.5cm}
	\end{flushright}
%\end{minipage}
\vspace{.25cm}
\hspace{-0.005cm}Winter Term 2019/2020

\vspace{0.25cm}

\begin{center}
	\vspace{.25cm}
	Dr.~Yannick Hoga \hspace{.5cm} Thilo Reinschlüssel \\
	\vspace{.25cm}
	\textbf{\Large{Multivariate Time Series Analysis}}\\
	\vspace{.25cm}
	\textbf{\large{Exercise Sheet 2}}\\
	\vspace{.125cm}
\end{center}




% body from markdown

\hypertarget{exercise-1-moments-and-simulation-of-a-var1-process}{%
\section{Exercise 1: Moments and Simulation of a VAR(1)
Process}\label{exercise-1-moments-and-simulation-of-a-var1-process}}

Take the model from Example 2.4 on Slide 2-6:

\begin{itemize}
  \item[a)] Derive a formula to obtain the population cross-covariance matrices for the lags 1 to 10 and compute them using R
  \begin{align*}
    z_t & = \phi_1 z_{t-1} + a_t\\
    \\
    \Gamma_0 & = \phi_1  \ \Gamma_0 \ \phi_1^{'} + \Sigma_a\\
    \Leftrightarrow \vec(\Gamma_0) &= (\phi_1 \otimes \phi_1) \cdot \vec(\Gamma_0) + \vec(\Sigma_a) \\
    \Leftrightarrow \vec(\Gamma_0) &= (I_{K^2} - \phi_1 \otimes \phi_1) \cdot \vec(\Sigma_a) \\
   \Rrightarrow \Gamma_1 & = \phi_1 \Gamma_0\\
   \Rrightarrow \Gamma_l & = \phi_{l-1} \Gamma_{l-1} = \phi_1^{l} \Gamma_0\\
  \end{align*}
\end{itemize}

\begin{Shaded}
\begin{Highlighting}[]
\CommentTok{# First define parameters/coefficients:}
\NormalTok{Phi <-}\StringTok{ }\KeywordTok{matrix}\NormalTok{(}\DataTypeTok{data =} \KeywordTok{c}\NormalTok{(}\FloatTok{0.2}\NormalTok{, }\FloatTok{-0.6}\NormalTok{, }\FloatTok{0.3}\NormalTok{, }\FloatTok{1.1}\NormalTok{), }
              \DataTypeTok{byrow =} \OtherTok{FALSE}\NormalTok{, }\DataTypeTok{nrow =} \DecValTok{2}\NormalTok{) }\CommentTok{# VAR coefficients}
\NormalTok{Sigma_a <-}\StringTok{ }\KeywordTok{matrix}\NormalTok{(}\DataTypeTok{data =} \KeywordTok{c}\NormalTok{(}\DecValTok{1}\NormalTok{, }\FloatTok{0.8}\NormalTok{, }\FloatTok{0.8}\NormalTok{, }\FloatTok{2.0}\NormalTok{), }
                  \DataTypeTok{byrow =} \OtherTok{FALSE}\NormalTok{, }\DataTypeTok{nrow =} \DecValTok{2}\NormalTok{) }\CommentTok{# innovations' covariances}

\NormalTok{Phi_kron <-}\StringTok{ }\KeywordTok{kronecker}\NormalTok{(}\DataTypeTok{X =}\NormalTok{ Phi, }\DataTypeTok{Y =}\NormalTok{ Phi) }\CommentTok{# kronecker product}
\CommentTok{# Phi %x% Phi # alternative command}

\NormalTok{Ident <-}\StringTok{ }\KeywordTok{diag}\NormalTok{(}\KeywordTok{ncol}\NormalTok{(Phi)}\OperatorTok{^}\DecValTok{2}\NormalTok{) }\CommentTok{# identity matrix with the same dimensions as Phi_kron}
\NormalTok{Gamma0.vec <-}\StringTok{ }\KeywordTok{solve}\NormalTok{(Ident }\OperatorTok{-}\StringTok{ }\NormalTok{Phi_kron) }\OperatorTok\StringTok{ }\KeywordTok{c}\NormalTok{(Sigma_a) }
\CommentTok{# c() works like the "vec" operator}

\NormalTok{Gamma0.mat <-}\StringTok{ }\KeywordTok{matrix}\NormalTok{(}\DataTypeTok{data =}\NormalTok{ Gamma0.vec, }\DataTypeTok{nrow =} \DecValTok{2}\NormalTok{, }\DataTypeTok{byrow =} \OtherTok{FALSE}\NormalTok{)}
\NormalTok{Gamma0.mat}
\end{Highlighting}
\end{Shaded}

\begin{verbatim}
##          [,1]     [,2]
## [1,] 2.288889 3.511111
## [2,] 3.511111 8.622222
\end{verbatim}

\begin{itemize}  
  \item[] \textit{Hint: A glance at the slides and a loop might save you some time}
  \item[b)] Based on your results, compute the cross-correlation matrices
  \item[c)] Draw a corresponding innovation sequence at for 300 periods from a (multivariate) Gaussian distribution and simulate the given VAR(1) process without any further built-in functions
  \item[] \textit{Hint: 'mvrnorm' and 'for' are still allowed}
  \item[d)] Plot the multivariate time series you have just created. Does it look stationary?
  \item[e)] Estimate the sample cross-covariance and cross-correlation matrices. Compare these with the population moment matrices from task a)
\end{itemize}

\hypertarget{exercise-2-checking-var1-stationarity}{%
\section{Exercise 2: Checking VAR(1)
Stationarity}\label{exercise-2-checking-var1-stationarity}}

Recall the conditions to check if a VAR(1) process is stationary. Now
assume the VAR(1) model \(z_t = \phi_1 z_{t-1} + a_t\) with \(a_t\) as
asequences of \(i.i.d.\) innovations:

\begin{itemize}
  \item[a)] Do you need to make further assumptions on the cross-correlations of $a_t$ to ensure stationarity
  \item[b)] Which of the following processes are stationary? $\phi_1 = \ldots$
  \begin{align*}
    \text{i)} 
    \begin{pmatrix}
      0.2 & 0.3 \\
      -0.6 & 1.1
    \end{pmatrix}
    \text{ii)} 
    \begin{pmatrix}
      0.5 & 0.3 \\
      0 & -0.3
    \end{pmatrix}
    \text{iii)} 
    \begin{pmatrix}
      1 & 0 \\
      0 & 1
    \end{pmatrix}
    \text{iv)} 
    \begin{pmatrix}
      1 & -1 \\
      1 & -1
    \end{pmatrix}
    \text{v)} 
    \begin{pmatrix}
      1 & -0.5 \\
      -0.5 & 0
    \end{pmatrix}
  \end{align*}
\end{itemize}

\end{document}
