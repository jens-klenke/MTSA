\documentclass[12pt,a4paper]{article}
\usepackage{lmodern}

\usepackage{scalerel}
\usepackage{bbm}
\usepackage{enumitem}
\usepackage{placeins}
\usepackage{amssymb,amsmath}
\usepackage{ifxetex,ifluatex}
\usepackage{fixltx2e} % provides \textsubscript
\ifnum 0\ifxetex 1\fi\ifluatex 1\fi=0 % if pdftex
  \usepackage[T1]{fontenc}
  \usepackage[utf8]{inputenc}
\else % if luatex or xelatex
  \ifxetex
    \usepackage{mathspec}
    \usepackage{xltxtra,xunicode}
  \else
    \usepackage{fontspec}
  \fi
  \defaultfontfeatures{Mapping=tex-text,Scale=MatchLowercase}
  \newcommand{\euro}{€}
\fi
% use upquote if available, for straight quotes in verbatim environments
\IfFileExists{upquote.sty}{\usepackage{upquote}}{}
% use microtype if available
\IfFileExists{microtype.sty}{%
\usepackage{microtype}
\UseMicrotypeSet[protrusion]{basicmath} % disable protrusion for tt fonts
}{}
\usepackage[lmargin = 2cm, rmargin = 2cm, tmargin = 2cm, bmargin = 2.5cm]{geometry}


% Figure Placement:
\usepackage{float}
\let\origfigure\figure
\let\endorigfigure\endfigure
\renewenvironment{figure}[1][2] {
    \expandafter\origfigure\expandafter[H]
} {
    \endorigfigure
}

%%%% Jens %%%%
\usepackage{titlesec}
\DeclareMathOperator*{\argmax}{arg\,max}
\DeclareMathOperator*{\argmin}{arg\,min}
\renewcommand{\vec}{\operatorname{vec}}
\newcommand{\tr}{\operatorname{tr}}
\newcommand{\Var}{\operatorname{Var}} % Variance
\newcommand{\MSE}{\operatorname{MSE}} % Variance
\newcommand{\VAR}{\operatorname{VAR}} % Vector autoregression
\newcommand{\Lag}{\operatorname{L}} % Lag operator
\newcommand{\Cov}{\operatorname{Cov}}
\newcommand{\diag}{\operatorname{diag}}
\newcommand{\adj}{\operatorname{adj}}
\newcommand{\loglik}{\operatorname{ll}}

\usepackage{centernot}

\allowdisplaybreaks

\titleformat{\section}
{\normalfont\large\bfseries}{\thesection}{1em}{}

\newcommand{\tmpsection}[1]{}
\let\tmpsection=\section
\renewcommand{\section}[1]{\tmpsection{\underline{#1}} }





%% citation setup
\usepackage{csquotes}

\usepackage[backend=biber, maxbibnames = 99, style = apa]{biblatex}
\setlength\bibitemsep{1.5\itemsep}
\addbibresource{R_packages.bib}
\usepackage{graphicx}
\makeatletter
\def\maxwidth{\ifdim\Gin@nat@width>\linewidth\linewidth\else\Gin@nat@width\fi}
\def\maxheight{\ifdim\Gin@nat@height>\textheight\textheight\else\Gin@nat@height\fi}
\makeatother
% Scale images if necessary, so that they will not overflow the page
% margins by default, and it is still possible to overwrite the defaults
% using explicit options in \includegraphics[width, height, ...]{}
\setkeys{Gin}{width=\maxwidth,height=\maxheight,keepaspectratio}
\ifxetex
  \usepackage[setpagesize=false, % page size defined by xetex
              unicode=false, % unicode breaks when used with xetex
              xetex]{hyperref}
\else
  \usepackage[unicode=true, linktocpage = TRUE]{hyperref}
\fi
\hypersetup{breaklinks=true,
            bookmarks=true,
            pdfauthor={Dr.~Yannick Hoga},
            pdftitle={Multivariate Time Series Analysis},
            colorlinks=true,
            citecolor=black,
            urlcolor=black,
            linkcolor=black,
            pdfborder={0 0 0}}
\urlstyle{same}  % don't use monospace font for urls
\setlength{\parindent}{0pt}
\setlength{\parskip}{6pt plus 2pt minus 1pt}
\setlength{\emergencystretch}{3em}  % prevent overfull lines
\setcounter{secnumdepth}{5}

%%% Use protect on footnotes to avoid problems with footnotes in titles
\let\rmarkdownfootnote\footnote%
\def\footnote{\protect\rmarkdownfootnote}

%%% Change title format to be more compact
\usepackage{titling}

% Create subtitle command for use in maketitle
\newcommand{\subtitle}[1]{
  \posttitle{
    \begin{center}\large#1\end{center}
    }
}

\setlength{\droptitle}{-2em}
  \title{Multivariate Time Series Analysis}
  \pretitle{\vspace{\droptitle}\centering\huge}
  \posttitle{\par}
\subtitle{Solution Exercise Sheet 9}
  \author{Dr.~Yannick Hoga}
  \preauthor{\centering\large\emph}
  \postauthor{\par}
  \date{}
  \predate{}\postdate{}


%% linespread settings

\usepackage{setspace}

\onehalfspacing


% Language Setup

\usepackage{ifthen}
\usepackage{iflang}
\usepackage[super]{nth}
\usepackage[ngerman, english]{babel}

%Acronyms
\usepackage[printonlyused, withpage, nohyperlinks]{acronym}
\usepackage{changepage}

% Multicols for the Title page
\usepackage{multicol}


% foot


\begin{document}

\selectlanguage{english}

%%%%%%%%%%%%%% Jens %%%%%
\numberwithin{equation}{section}




\restoregeometry


%%% Header 

\begin{minipage}{0.6\textwidth}
University of Duisburg-Essen\\
Faculty of Business Administration and Economics\\
Chair of Econometrics\\
\end{minipage}

%\begin{minipage}{0.4\textwidth}
	\begin{flushright}
	\vspace{-3cm}
	\includegraphics*[width=5cm]{../Includes/duelogo_en.png}\\
	\vspace{.125cm}
	\end{flushright}
%\end{minipage}
%\vspace{.125cm}
\hspace{-0.005cm}Winter Term 2019/2020

\vspace{0.05cm}

\begin{center}
	\vspace{.25cm}
	Dr.~Yannick Hoga \hspace{.5cm} Thilo Reinschlüssel \\
	\vspace{.25cm}
	\textbf{\Large{Multivariate Time Series Analysis}}\\
	\vspace{.25cm}
	\textbf{\large{Solution Exercise Sheet 9}}\\
	\vspace{.125cm}
\end{center}




% body from markdown

\hypertarget{exercise-1-granger-causality-theory}{%
\section{Exercise 1: Granger Causality --
Theory}\label{exercise-1-granger-causality-theory}}

Let \(z_t = (x_t, y_t)^{'}\) be a stationary time series with two
dimensions. Define the forecast errar as the univariare series
\(e_T (h) = y_{T +h} - y_T (h)\) with
\(y_T (h) = \mathbb{E} \left( y_{T + h} | \Omega_T \right)\). The
information set \(\Omega_T\) contains all relevant variables available
whereas
\(\Omega^{\setminus x}_{T} = \Omega_T \setminus \left\{ x_t\right\}_{t = 0}^{T}\)
omits the variable \(x\) entirely. (This setting is the univariate
equivalent to definition 6.1 on Slide 6-4.)

\begin{itemize}
  \item[a)] Prove that $\mathbb{E} \left( e_T(h) | \Omega_{T}^{\setminus x} \right) = 0$.
\end{itemize}

\emph{Solution:}

\begin{align*}
  z_t & = \begin{pmatrix} x_t \\ y_t \end{pmatrix}\\
  \\
  \mathbb{E} \left( e_T (h) | \Omega_{T}^{\setminus x} \right) & = \mathbb{E} \left( y_{T + h} - \mathbb{E} \left( y_{T +h} | \Omega_T \right) | \ \Omega_{T}^{\setminus x} \right)\\
  & \overset{\text{LIE}}{=} \mathbb{E} \left( \mathbb{E} \left( y_{T +h} - y_{T + h}| \Omega_T \right) | \Omega_{T}^{\setminus x} \right)\\
  & \text{since } \Omega_{T}^{\setminus x} \subseteq \Omega_{T}  \\
  & \text{LIE = Law of Iterated Expectations}\\
  & = 0\\
\end{align*}

\begin{itemize}
  \item[b)] Prove that $\Var \left(e_t (h)| \Omega_T \right) \leq \Var \left(e_t (h)| \Omega_T^{\setminus x} \right)$
\end{itemize}

\emph{Solution:}

2 Theorems necessary for the proof:

1 Conditional Jensen's Inequality

\(g(\cdot ): \mathbb{R}^{m} \rightarrow \mathbb{R}\) is convex (like
\(\chi^2\)), then for any random vectors (y,x) for which
\(\mathbb{E} (||y||) < \infty\) and \(\mathbb{E} (|| g(y)||)< \infty\),
\(g \left( \mathbb{E} (y|x) \right) \leq \mathbb{E} \left( (g(y)|x) \right)\).
It is the other way around for concave functions.

2 Conditioning Theorem

If \(\mathbb{E}(|y|) < \infty\), then
\(\mathbb{E}(g(x) y | x) = g(x) \cdot \mathbb{E}(y|x)\). If in addition
\(\mathbb{E} (|g(x) y|) < \infty\), then
\(\mathbb{E} (g(x) y) = \mathbb{E} (g(x) \mathbb{E} (y|x))\).

Back to Granger:

\(e_T (h) = y_{T +h} - y_T (h)\) is a scalar. We know that
\(\mathbb{E} \left( e_T (h) | \Omega_T^{ \setminus x} \right) = 0\),
\(\mathbb{E} \left( e_T (h) | \Omega_T \right) = 0\) and
\(\Var(e_T (h)) < \infty\) since \(y_t\) is a weakly stationary (w.s.)
process. Furthermore, w.s. implies that
\(\mathbb{E} (y_t) < \infty, \mathbb{E} (y_t^2) < \infty\).

From Jensen's Inequality it follows:

\begin{align*}
  \left[ \mathbb{E} \left( y_{T + h} | \Omega_{T}^{\setminus x } \right) \right]^2 & \overset{\text{LIE}}{=} \left[ \mathbb{E} \left[ \mathbb{E} \left( y_{T + h} | \Omega_{T} \right) | \Omega_T^{\setminus x} \right] \right]^2 \\
  & \leq \mathbb{E} \left[ \left[ \mathbb{E} (y_{t +h} | \Omega_T)\right]^2 | \Omega_{t}^{\setminus x} \right]
\end{align*}

Taking conditional expactations:

\begin{align*}
  \mathbb{E} \left[ \left( \mathbb{E}\left[ y_{T + h | \Omega_{T}^{\setminus x}} \right]  \right)^2 \right] \leq \mathbb{E} \left( \left[ \mathbb{E} \left( y_{T +h} | \Omega_T \right) \right]^2 \right)
\end{align*}

This extends to:

\begin{align*}
  \left[ \mathbb{E} (y_{T + h}) \right]^2 & \leq \mathbb{E}  \left( \left[ \mathbb{E} \left( y_{T + h} | \Omega_T^{\setminus x} \right) \right]^2 \right)\\
  \text{Since} \; \mathbb{E} (y_{T + h}) & = \mathbb{E} \left[ \mathbb{E} \left( y_{T +h} | \Omega_{T}^{\setminus x}\right) \right]
\end{align*}

\end{document}
